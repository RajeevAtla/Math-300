\documentclass{article}
\usepackage[sans, stdmargin, noindent]{../rajeev}

\begin{document}


\problem{}
There is an error in Lemma 8.7, similar to the last time I pointed out my own error.
Find it, fix the statement of the lemma, and amend the proof to adapt to this.
Discuss in Discord.

\hrule

The main error is that the infinite sequence can be made up of empty sequences.
The proof doesn't account for that, and an additional argument needs to be appended in order for it to make sense.

\problem{}

\subproblema{}

Show that there is no injection from $\brak{n+1}$ to $\brak{n}$ for any integer.
\hrule

For the sake of notation, let $A = \brak{n+1}$ and $B = \brak{n}$.
We consider the function $f : A \to B$.
Since $f$ is a function, it must map all elements of $A$ to some elements in $B$.
However, each element of $A$ can't map to a distinct element of $B$ because $\abs{A} > \abs{B}$.
At minimum, two distinct elements $x, y \in A$ map to the same element of $B$.
Therefore, the function cannot be an injection.

\problem{}

Exhibit a bijection from $\ZZ$ to $\QQ$.

\hrule

Because they are of the same cardinality, there exists a bijection between $\QQ^-$ and $\ZZ^-$.
Since $\QQ = \QQ^- \cup \QQ^+$ and $\ZZ = \NN \cup \ZZ^- \cup \set{0}$, the cardinality of these two sets are equal, as they are both countably infinite.
Therefore, there exists a bijection between the two.

\problem{}

Show that if a set $A$ has a bijection to a subset of $\NN$, then it is countable.
Conclude that if $\abs{A} \leq \abs{\NN}$, then it is countable.

\hrule

Pick some $B \subseteq \NN$ and define the bijection $f : A \to B$.
Since $f$ is a bijection, $\abs{A} = \abs{B}$.
Since $B$ is a subset of $\NN$, $\abs{B} \leq \abs{\NN}$.
Therefore, $A \leq \abs{\NN}$.
If the cardinality of $A$ is the same as that of $\NN$, then it is countably infinite.
If it is less than the cardinality of $\NN$, then it is a finite set.
In either case $A$ is a countable set.

\problem{}

\subproblema{}
Suppose $A_1, A_2$ are countable sets.
Show that there is an injection $f : A_1 \times A_2 \to \NN$.
[Hint: Fundamental theorem of Arithmetic.]

\hrule

For the Cartesian product to be valid, we must have that $\abs{A}_1 = \abs{A}_2$.
In addition, the cardinality of the Cartesian product is equal to that of its input sets.
Since both $A_1$ and $A_2$ are countable, $A_1 \times A_2$ will also be countable.
Therefore, $\abs{A_1 \times A_2} \leq \abs{\NN}$.
Then, there must exist an injection between $A_1 \times A_2$ and $\NN$.

\subproblema{}
Generalize to any countable $A_1 \times A_2 \times \dots \times A_n$.

\hrule

The essence of the argument still holds: $\abs{A_1} = \abs{A_2} = \dots = \abs{A_n} = \abs{A_1 \times A_2 \times \dots \times A_n}$.
This cardinality will then be less than or equal to that of the natural numbers.

\problem{}

\subproblema{}
Suppose that $\mathcal{A}$ is a countable collection of countably infinite sets.
That is, there are countably many sets contained in $\mathcal{A}$.
Show that $\bigcup \limits_{A \in \mathcal{A}} A$ is countable.

\hrule

Since the collection of sets is countable, there is an injection with the natural numbers as a result.
Let $\mathcal{A}_i$ be the $i$-th element set of $\mathcal{A}$.
Similarly, let $\mathcal{A}_{ij}$ be the $j$-th element of the $i$-th element set of $\mathcal{A}$.
We can write out each elements in a sparse $n$-by-$\NN$ grid, where $n \leq \abs{\NN}$.
Using the results of Problem 5, we know that $\abs{\NN \times \NN} = \abs{\NN}$, so the cardinality of the joint union is therefore less than or equal to that of the natural numbers.
The joint union is therefore countable.

\subproblema{}
Deduce that the collection of all finite subsets of $\NN$ is countable.
Suppose $\NN_k$ is the set of all finite subsets of $\NN$ with cardinality $k$.



\end{document}