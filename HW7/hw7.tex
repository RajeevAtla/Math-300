\documentclass{article}
\usepackage[sans, stdmargin, noindent]{../rajeev}

\begin{document}


\problem{}
Let $A = \set{1, 2}, B = \set{x, y}$. List all functions from $A \to B$ that are:

\subproblema{}
injections;
\hrule

By the definition of an injection, each element of $B$ may have at most $1$ element of $A$ that maps to it.
The possible functions are then,

\begin{align*}
  & \set{\pars{1, x}, \pars{2, y}} \\
  & \set{\pars{2, x}, \pars{1, y}} \\
\end{align*}

\subproblema{}
surjections;
\hrule

By the definition of a surjection, each element of $B$ must be mapped to by an element of $A$ at least once.
The possible functions are then,

\begin{align*}
  & \set{\pars{1, x}, \pars{2, y}} \\
  & \set{\pars{2, x}, \pars{1, y}} \\
\end{align*}


\subproblema{}
bijections;
\hrule

A function is only a bijection if it is both an injection and a surjection.
The possible functions are then,

\begin{align*}
  & \set{\pars{1, x}, \pars{2, y}} \\
  & \set{\pars{2, x}, \pars{1, y}} \\
\end{align*}

\subproblema{}
none of the above.
\hrule

The other functions that exist to map these two sets are

\begin{align*}
  & \set{\pars{1, x}, \pars{2, x}} \\
  & \set{\pars{1, y}, \pars{2, y}} \\
\end{align*}

\problem{}
\subproblema{}
Supply a proof for Proposition 7.8.

\textbf{Proposition 7.8.}
If $f : A \to B$ and $g : B \to C$ are both injective, then their composition $g \circ f : A \to C$ is injective.

\hrule

Consider two arbitrary elements $x, y \in A$ such that $g \pars{f \pars{x}} = g \pars{f \pars{y}}$.
Since $g$ is an injection, this means $f \pars{x} = f \pars{y}$.
Similarly, since $f$ is also an injection, this means $x = y$.
This means that $g \circ f$ is also an injection.

\subproblema{}
Suppose $f:A\to B$ and $g:B\to C$.
If $g\circ f$ is injective, is it necessarily the case that $f$ is injective?
Is it necessarily the case that $g$ is injective?
Prove or disprove your claims.

\hrule

We show that $f$ is necessarily injective.
Pick some arbitrary $x, y \in A$ such that $f \pars{x} = f \pars{y}$.
This is equivalent to $g \pars{f \pars{x}} = g \pars{f \pars{x}}$.
Since $g \circ f$ is injective, this implies that $x= y$.
Therefore, $f$ is injective.

Note that $g$ doesn't need to be injective we can't go from $g(x) = g(y)$ to $g(f(x)) = g(f(y))$ without changing sets.

\subproblema{}

Repeat part (b) but replace injective with surjective.
\hrule

We show that $g$ is necessarily surjective.
Choose $z \in C$.
We know that there exists $x \in A$ such that $g \circ f \pars{x} = z$ since $g \circ f$ is surjective.
Define $y := f \pars{x}$.
Then there exists a $y \in B$ such that $g \pars{y} = z$, so $g$ is surjective.

We can't show the same for $f$ since it is already inside another function, so we can't make the same argument.

\problem{}

Find a function from $\RR$ to $\RR$, and supply a proof of your claim, that is:

\subproblema{}

an injection but not a surjection;

\hrule

Consider $f(x) = e^x$.
It is an injection because each $x$-value results in a different output.
It is not a surjection because non-positive real numbers aren't achieved no matter what the value of $x$ is.

\subproblema{}

a surjection but not an injection;

\hrule

Consider $f(x) = x \sin x$.
This function is a surjection because all real numbers are achieved given a suitable real number $x$.
However, this function is not an injection because there are $f(x)$-values that are produced by more than one $x$-value.

\subproblema{}

a bijection;

\hrule

Consider $f(x) = x$.
This function is an injection because each value of $f(x)$ is produced by a distinct $x$-value.
Similarly, this function is a surjection because all real numbers are achieved given a suitable real number $x$.
Since the function is both an injection and a surjection, it is a bijection.

\subproblema{}

neither a surjection nor an injection.

\hrule

Consider $f(x) = x^2$.
This function is not an injection because multiple $x$ values can map to each $f(x)$-value.
This function is not a surjection because the negative real numbers cannot be achieved for any real $x$.

\problem{}
Let $f : A \to B$, and let $G_1, G_2 \subseteq A$, and let $H_1, H_2 \subseteq B$.
\subproblema{}
Is it true that $f^{-1} \pars{f \pars{G_1}} = G_1$.
If so, prove it; if not, provide a counterexample, and provide the correct relation between the two sets, and justify your answer.
\hrule
The image of $G_1$, $f \pars{G_1}$ will be some subset of $B$, which we denote by $B_1$.
Unfortunately, we don't know whether or not the function is injective, so the pre-image $f^{-1} \pars{B_1}$ may contain extra elements.
For a counterexample, consider the function $f \pars{x^2}$ over the subset $\brak{-2, -1}$.
The final set that is returned will be $\brak{-2, -1} \cup \brak{1, 2}$.

The correct relation is $f^{-1} \pars{f \pars{G_1}} \subseteq G_1$.

\subproblema{}
Is it true that $f \pars{f^{-1} \pars{H_1}} = H_1$.
If so, prove it; if not, provide a counterexample, and provide the correct relation between the two sets, and justify your answer.
\hrule

Let the pre-image of $H_1$ be $A_1$.
Since $f$ is a function, by definition, each element of $A_1$ is mapped to its appropriate element of $H_1$.
Therefore, $f \pars{A_1} = H_1$, proving this statement to be true.

\subproblema{}
As a counterexample, we choose $f \pars{x} = x^2$ and the intervals $ \brak{-1,0}$ and $\brak{0, 1}$.



\end{document}