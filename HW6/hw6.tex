\documentclass{article}
\usepackage[sans, stdmargin, noindent]{../rajeev}

\begin{document}

\problem{}

Consider $\brak{5}=\set{1,2,\dots,5}$.
Give an example of, or show that such a request impossible, a nonempty relation on $\brak{5}$ that is:



\subproblema{}
not symmetric, but reflexive and transitive.
\hrule

We present the following relation:
$\sim = \set{\pars{a, b} |\ a \text{ divides } b \text{ and } a, b \in \brak{5}}$.

This is not symmetric because if $a$ divides $b$, then it is not necessarily true that $b$ divides $a$.
This is reflexive because every integer divides itself.

If $a$ divides $b$, then $a$ must be a factor of $b$.
Similarly, if $b$ divides $c$, then $b$, which has $a$ as a factor, is a factor of $c$.
Therefore, $a$ is a factor of $c$, so $a$ divides $c$.
This means the relation is transitive.

\subproblema{}
not transitive, but reflexive and symmetric.
\hrule

We present the following relation:
$\sim = \set{\pars{a, b} |\, \abs{a-b} \leq 1}$.

This is reflexive because $ \abs{a-a} = 0 \leq 1$.
This is symmetric because $ \abs{a-b} = \abs{b-a}$, so if $\abs{a-b} \leq 1$, then $\abs{b-a} \leq 1$.

Suppose we have some $a, b, c \in \brak{5}$.
If $a \sim b$, then $a \in \set{b + 1, b, b - 1}$.
If $b \sim c$, then $b \in \set{c + 1, c, c - 1}$.
This means that $a \in \set{c+2, c+1, c, c-1, c-2}$.
This means that $a$ and $c$ may satisfy the relation, but they don't have to.
Therefore, the relation isn't transitive.


\subproblema{}
not reflexive, but symmetric and transitive.
\hrule

Such a relation isn't possible, and we proceed by proof by contradiction.
Suppose such a relation could be found, and denote it by $\sim$.
Since the relation is symmetric, if $x \sim y$, then $y \sim x$.
Since the relation is transitive, $x \sim y$ and $y \sim x$ implies $x \sim x$.
However, the relation isn't reflexive, so this is a contradiction.
Therefore, such a relation can't exist.

\subproblema{}
both symmetric and antisymmetric.
\hrule

Consider the following relation: $\sim = \set{\pars{a, b} |\, a = b}$.

This is symmetric because if $a \sim b$, then $a = b$, so therefore $b \sim a$.
Moreover, this is antisymmetric, because for all $a \neq b$, the relation isn't true from either side, as explicitly defined in its construction.

\problem{}
Determine if each of the following relation on a set A is an equivalence relation or not.
If so, exhibit the equivalence classes.
Justify each answer.

\subproblema{}
$A=\mathbb{R}^2$, $(a,b) R(c,d)$ if $a^2+b^2=c^2+d^2$
\hrule

Pick some $\pars{a, b}$, $\pars{c, d}$, and $\pars{e, f}$ in $\RR$.
$a^2 + b^2 = a^2 + b^2$, so the relation is reflexive.
Moreover, if $a^2 + b^2 = c^2 + d^2$, then $c^2 + d^2 = a^2 + b^2$, so the relation is symmetric.
If $a^2 + b^2 = c^2 + d^2$ and $c^2 + d^2 = e^2 + f^2$, then $a^2 + b^2 = e^2 + f^2$, so the relation is transitive.
Since the relation is reflexive, symmetric, and transitive, the relation is an equivalence relation.

This relation can be thought of as the set of points in $\RR^2$ that form a circle with a center at $\pars{0, 0}$.
Therefore, the equivalence classes can be written in the form of $\set{ \pars{x, y} | x^2 + y^2 = k^2}$, where $k \in \RR$.

\subproblema{}

$A = \QQ, R = \varnothing$ the empty relation.
\hrule

Under this relation, no two rational numbers are related.
Therefore, the relation is symmetric, reflexive, and transitive by vacuous proof.

There are no resulting equivalence classes because the relation does not relate any rational numbers at all.

\problem{}

Prove or disprove the converse of Proposition 6.9: that is, a partition $\mathcal{A}$ on $A$ induces an equivalence relation on $A$ by $x\sim y$ if and only if there exists some $B\in\mathcal{A}$ such that $B$ contains both $x$ and $y$.

\hrule
Suppose that $B = \mathcal{A}_a$, so that for all $x \in B$, $x \sim a$.
By the transitive property, which $\sim$ obeys since it is an equivalence relation, if $y$ is also in $B$, then $x \sim y$.


\problem{}
Consider the relation $D\coloneqq \{(a,b)\in \mathbb{N}\times \mathbb{N}\mid a\text{ divides }b\}$. Show that $D$ is a partial ordering.
\hrule

Any number divides itself so $D$ is reflexive.
If $a$ divides $b$, then $a$ must be a factor of $b$.
Similarly, if $b$ divides $c$, then $b$, which has $a$ as a factor, is a factor of $c$.
Therefore, $a$ is a factor of $c$, so $a$ divides $c$.
This means the relation is transitive.
If $a$ divides $b$, then that does not necessarily mean that $b$ divides $a$.
Therefore, the relation is anti-symmetric.
Since the relation is reflexive, anti-symmetric, and transitive, it is a partial ordering.


\problem{}

This problem will deal with what we call a total ordering.


\textbf{Definition 6.16.}
A \textbf{total ordering} or a \textbf{linear ordering} is a partial ordering such that any two elements are comparable; that is, for any $a$ and $b$, either $a \prec b$ or $b \prec a$. 

Suppose $A$ and $B$ are two sets, with total orderings $\prec_A$ and $\prec_B$ respectively.
Define 
\[
    \prec_L\, := \{((a, b), (c, d)) \mid (a \neq c \land a \prec_A c) \lor (a = c \land b \prec_B d) \},
\]
and
\[
    \prec_P\, := \{((a, b), (c, d)) \mid a \prec_A c \land b \prec_B d \}.
\]

\textbf{Note 6.17.}
$\prec_L$ is known as the lexicographic ordering, and $\prec_P$ is known as the product ordering.

\subproblema{}
Describe in your own words how $\prec_L$ and $\prec_P$ work.
\hrule

The lexicographic ordering first compares the first value in the pair if they are not equal.
Otherwise, it compares the second value in the pair.

The product ordering checks both coordinates and only works if both of them are less than their counterparts.

\subproblema{}
Show that $\prec_P$ is a partial ordering.
\hrule

Since both $\prec_A$ and $\prec_A$ are partial orderings, both are anti-symmetric, reflexive, and transitive.
Therefore $\prec_P$ is also anti-symmetric, reflexive, and transitive because it is the logical disjunction of two partial orderings.
This makes it also a partial ordering.

\subproblema{}
Show that $\prec_P \subseteq \prec_L$.
\hrule

Consider an arbitrary element $x = \pars{\pars{a, b}, \pars{c, d}} \in \prec_P$.
Since this element is a member of $\prec_P$, $a \prec_A c$ and $a \prec_B b$.
Because of this, $x$ also satisfies the criteria to be an element of $\prec_L$.
Since both the lexicographic and product orderings are defined over the same space, $x$ is also in $\prec_L$.
This works for any arbitrary element, so $\prec_P \subseteq \prec_L$.


\end{document}