\documentclass{article}
\usepackage[sans, stdmargin, noindent]{../rajeev}

\begin{document}

\problem

For this problem, we define $H$ to be all members of the species \emph{Homo sapiens} (humans) currently living, $D$ as all members of the species \emph{Canis familiaris} (domesticated dogs) currently living. Let $L\pars{x, y} := $ \emph{$x$ loves $y$}.

\subproblema
$$
\forall x \in H,\ \forall y \in D,\ L \pars{x, y}
$$

\subproblema
$$
\forall x \in H,\ \exists y \in D : L \pars{x, y} \lor x \in D
$$

\subproblema
Define $z \pars{x} := $ \emph{$x$ is the greatest integer}.
$$
\forall x \in \ZZ,\ \lnot z \pars{x}
$$

\subproblema
Define $R \pars{x} := $ \emph{$x$ needs rest to be healthy}, $E \pars{x} := $ \emph{$x$ needs excercise to be healthy}, and $D \pars{x} := $ \emph{$x$ needs a good diet to be healthy}.

$$
\forall x \in H,\ R \pars{x} \land E\pars{x} \land D\pars{x}
$$

\subproblema
$$
\forall x \in \ZZ,\ \exists y \in \ZZ : y > x
$$

\subproblema
Define $D \pars{x, y} := $ \emph{$x$ divides $y$}.
Define $S := \ZZ^+ \setminus \set{1, x} $ as the positive integers without $1$ and $x$.

$$
\not \exists y \in S : D \pars{x, y}
$$

\problem

All the definitions from the previous problem carry over as appropriate.

\subproblema

$$
\exists x \in H : \exists y \in D : \lnot L \pars{x, y}
$$


\subproblema

$$
\exists x \in H : \forall y \in D, \lnot L \pars{x, y} \land x \not \in D
$$

\subproblema

$$
\exists x \in \ZZ : z \pars{x}
$$

\subproblema

$$
\exists x \in H : \lnot R \pars{x} \lor \lnot E \pars{x} \lor \lnot D \pars{x}
$$

\subproblema

$$
\exists x \in \ZZ : \forall y \in \ZZ,\ y \leq x
$$

\subproblema

$$
\exists y \in S : \lnot D \pars{x, y}
$$


\problem

\subproblemat{Consistency}

\begin{align*}
  \pars{p \lor \lnot p} & \iff T \\
  \lnot \pars{p \lor \lnot p} & \iff \lnot T \\
  \lnot p \land p & \iff F \\
\end{align*}

\subproblemat{Absorption}

\begin{align*}
  \pars{p \land \pars{p \lor q}} & \iff \pars{\pars{p \land p} \lor \pars{p \land q}} \\
                                 & \iff \pars{p \lor \pars{p \land q}}
\end{align*}

\subproblemat{Contrapositive}

\begin{align*}
  \pars{p \Rightarrow q} & \iff \pars{\lnot p \lor q} \\
                         & \iff \pars{q \lor \lnot p} \\
                         & \iff \pars{\lnot \pars{\lnot q} \lor \lnot p } \\
                         & \iff \pars{\lnot q \Rightarrow \lnot p} \\
\end{align*}


\problem

$$
P \ovee Q \equiv \pars{P \lor Q} \land \pars{\lnot \pars{P \land Q}}
$$

Intuitively, this makes sense, because it explicitly checks that one of $P$ or $Q$ are true, but also explicitly checks that both aren't true.

\problem

\subproblema

For all $x$ in the set of real numbers, there exists a $y$ in the set of real numbers such that $x$ and $y$ sum to $0$.

This is true because for each $x \in \RR$, there's a $-x$, which is its additive inverse, and by definition, the two sum to 0.

\subproblema

There exists an $x$ in the set of real numbers such that for all $y$ in the set of real numbers $x$ and $y$ sum to $0$.

This is false, because not every real number is the additive inverse to $x$.

\problem

The key issue here is that we need to specify that \textbf{exactly one} element of $X$ satisfies $P\pars{x}$, so we say that there's exists an element in the set that satisfies $P\pars{x}$ and if another value satisfies $P\pars{x}$ then it is equal to the original value.

$$
\exists ! x \in X : P\pars{x} \equiv \exists x \in X : \pars{P\pars{x} \land \forall y \in X, \pars{P\pars{y} \Rightarrow \pars{y=x}}}
$$

\end{document}
