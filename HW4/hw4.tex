\documentclass{article}
\usepackage[sans, stdmargin, noindent]{../rajeev}

\begin{document}

\problem{}
Using induction, prove that \[\sum_{k=1}^n k^2=\frac{n(n+1)(2n+1)}{6}.\]
\hrule

We proceed using induction.
We first show that this holds true for $n=2$.

\begin{align*}
  \sum \limits_{k=1}^{k=2} k^2 &= 1^2 + 2^2 \\
                               &= 1 + 4 \\
                               &= 5 \\
  \frac{n \pars{n+1} \pars{2n+1}}{6} &= \frac{2 \pars{3} \pars{5}}{6} \\
                               &= 5 \\
\end{align*}

We then show that this holds for $n+1$ assuming that it holds for $n$.

\begin{align*}
  \sum \limits_{k=1}^{n+1} k^2 &= \pars{n+1}^2 + \sum \limits_{k=1}^{n} k^2 \\
                               &= \pars{n+1}^2 + \frac{n \pars{n+1} \pars{2n+1}}{6} \\
                               &= \pars{n+1} \pars{n+1 + \frac{n \pars{2n+1}}{6}} \\
                               &= \pars{n+1} \pars{\frac{n^2}{3} + \frac{7n}{6} + 1} \\
                               &= \frac{1}{6} \pars{n+1} \pars{2n^2 + 7n + 6} \\
                               &= \frac{1}{6} \pars{n+1} \pars{2n+3} \pars{n+2} \\
                               &= \frac{1}{6} \pars{n+1} \pars{2 \pars{n+1} + 1} \pars{\pars{n+1} + 1}
\end{align*}

\problem
Using induction, show that for all $n \in \NN$, $8$ divides $5^{2n} - 1$
\hrule
We proceed using induction.
We first show that this holds for $n=1$.
$$5^2 - 1 = 24$$
This is divisible by $8$.

We then assume that this holds true for $n$ and proceed to show that it holds true for $n+1$.
Using some clever algebraic manipulation, we can write the expression for $n+1$ as
\begin{align*}
  5^{2 \pars{n+1}} - 1 &= 25 \cdot 5^{2n} - 1 \\
                       &= 25 \pars{5^{2n} - 1} + 25 \pars{1} - 1 \\
                       &= 25 \pars{5^{2n} - 1} + \pars{5^2 - 1} \\
\end{align*}

Both terms of the expression are divisible by 8, so the whole expression must also be divisible by 8, using Euclidean division.

\problem{}
Suppose there are $n$ people in a room.
Everyone in the room is very friendly, as it happens, and each person wants to shake every other person's hand.
Show that there are $\frac{n^2-n}{2}$ handshakes that will occur.
(For example, for $n=2$, there will be one handshake that occurs between the two people).
\hrule

We proceed using induction.
We first show that this holds for $n=3$.
The first person shakes hands with the second and third.
The second person shakes hands with the third.
That is a total of $3$ handshakes, which is consistent with the formula: $\frac{3^2 - 3}{2} = 3$.

We then assume that this formula holds for $n$ people and consider what happens when it is applied to $n+1$ people.
This is analogous to adding one extra person into the mix of people who need to shake hands.
This person only needs to shake everyone else's hands once, adding $n$ handshakes.
The result is then
\begin{align*}
  n + \frac{n \pars{n-1}}{2} &= n + \frac{n^2 - n}{2} \\
                             &= \frac{n^2 + n}{2} \\
                             &= \frac{\pars{n+1} \pars{n}}{2} \\
                             &= \frac{\pars{n+1} \pars{\pars{n+1}-1}}{2}
\end{align*}

\problem
\textbf{(Towers of Hanoi).} Consider the following figure, where we have three rods, and 5 disks stacked on top of each other on the leftmost rod, and the largest disk is on the bottom, and the smallest is on top.
The objective of the puzzle is to move all the disks to the rightmost rod, while observing the following three rules:
\begin{itemize}
\item Only one disk may be moved at a time. 
\item A ``move" is taking the topmost disk from one of the rods and moving it to a different rod.
\item A larger disk may not be placed on a smaller disk. 
\end{itemize}
Now consider the same game with $n$ disks instead of 5. Show, using induction, that the puzzle may be solved in $2^n-1$ moves.
\hrule

We proceed using induction.
Consider the case when $n=1$.
Then, $2^1 - 1 = 1$, so this works, as only one step is needed to move one disk to the rightmost rod.

We then assume this works for $n$ disks and consider what happens with $n+1$ disks.
Let the $n+1$th disk be the bottom-most disk in the original stack.
It takes $2^n - 1$ steps to move the top $n$ disks to the middle stack.
We then move the $k+1$ disk to another stack with $1$ additional move.
Finally, combining all the existing stacks together takes another $2^n - 1$ steps.
This adds up to

\begin{align*}
  \pars{2^n - 1} + \pars{1} + \pars{2^n - 1} &= 2 \cdot 2^n - 1 \\
                                             &= 2^{n+1} - 1
\end{align*}

\problem
Show that for any convex polygon with $n$ vertices, the sum of interior angles is $\pars{n-2}\pi$. A convex polygon is a polygon where for any two points in the interior, the line segment that joins them is contained within the polygon. However, you may use the easier definition, which is that a polygon is convex when all interior angles are less than $\pi$.
\hrule

As a preliminary, a convex polygon that has $n$ vertices will have $n$ sides.

We proceed using induction.
For a triangle, it is well known that the sum of the interior angles is $\pi$, showing that the statement is true for $n=3$.

We then assume the statement holds for $n$ vertices, and show that it holds for $n+1$ vertices.
Consider an arbitrary $n+1$-sided convex polygon.
This shape can be decomposed into a triangle and a $n$-sided convex polygon by drawing a line through two distinct vertices that share a common adjacent vertex.
This decomposition is valid because the line drawn will lie completely in the polygon in a convex polygon, by definition.
Then its interior angles will be the sum of those of the triangle and the $n$-sided polygon, or
\begin{align*}
  \pi + \pars{n-2} \pi &= \pars{n-1} \pi \\
                       &= \pars{\pars{n+1}-2} \pi \\
\end{align*}

\problem{}

The Computer Science department considers induction the most important part of Math 300 (even though it \textit{definitely} is not), and the reason why this is the case is that induction provides a way to prove that an algorithm is correct.
Consider the following \texttt{Adamsort} algorithm, which takes an array of length $n$ as input and outputs an array which contains the same elements, but sorted.
\begin{algorithm}\caption{Adamsort}\begin{algorithmic}
    \Require $A$ an array of length $n\ge 0$
    \Ensure $A$ is sorted at the end. That is, $i<j$, then $A[i]\le A[j]$.
    \For{$1\le i\le \binom{n}{2}$} 
    \If {there exist $j<k$ such that $A[j]>A[k]$} 
    \State Swap $A[j]$ and $A[k]$
    \Else \State Break 
    \EndIf
    \EndFor
    
  \end{algorithmic}\end{algorithm}
    
Prove using induction that this algorithm converges to a sorted list.
\hrule
We proceed using induction.
Consider the case when $n=2$ with an array $\brak{a_0, a_1}$ with arbitrary elements.
There are two cases:
\begin{enumerate}
\item If $a_0 \leq a_1$, then the array is already sorted, so the algorithm does nothing.
\item If $a_0 > a_1$, then $a_0$ and $a_1$ are swapped, making the array sorted.
\end{enumerate}
Therefore, no matter the values of the array values, the algorithm works for $n=2$.

We then assume the algorithm works for $n$ and prove that it works for $n+1$.
Consider an array of length $n+1$.
We then place aside the last element, $a_n$ and only consider the first $n$ elements, which we can use Adamsort to convert into a sorted array.
We are then left with a sorted array of $n$ elements and $a_n$, which we append to the already sorted array.
We then use Adamsort to sort the resultant array.
There are two main cases:
\begin{enumerate}
\item If $a_n$ is greater than all of the other elements, then the array is already sorted.
\item If there exists an element $a_i$ with minimal index $i$ such that $a_i > a_n$, then the elements $a_i$ and $a_n$ will switch positions.
\end{enumerate}
These cases are checked continually until the algorithm converges.

\problem{}

\subproblema{}

\textbf{(Pascal's Identity).}
The \textbf{binomial coefficient} is written \[\binom{n}{k}:=\frac{n!}{k!(n-k)!}\] (\verb|\binom{n}{k}|) and is the coefficient on $x^k$ of the expansion of the binomial $(1+x)^n$.
It is also the number of ways to choose $k$ elements from a set with $n$ elements.
Prove that \[\binom{n}{k}=\binom{n-1}{k}+\binom{n-1}{k-1}\]
[Hint: no need to use induction here, just verify using computation.]
\hrule
We proceed by using the definition of a binary coefficient (which was provided), a lot of algebraic manipulation, and occasionally using the fact that $n! = n \pars{n-1}!$.

\begin{align*}
  \binom{n-1}{k} + \binom{n-1}{k-1} &= \frac{\pars{n-1}!}{k! \pars{n-1-k}!} + \frac{\pars{n-1}!}{\pars{k-1}! \pars{n-k}!} \\
                                    &= \frac{\pars{n-1}!}{k \pars{k-1}! \pars{n-1-k}!} + \frac{\pars{n-1}!}{\pars{k-1}! \pars{n-k}!} \\
                                    &= \frac{\pars{n-1}!}{k \pars{k-1}! \pars{n-1-k}!} + \frac{\pars{n-1}!}{\pars{n-k}\pars{k-1}! \pars{n-1-k}!} \\
                                    &= \frac{\pars{n-1}!}{\pars{k-1}! \pars{n-1-k}!} \pars{\frac{1}{k} + \frac{1}{n-k}} \\
                                    &= \frac{\pars{n-1}!}{\pars{k-1}! \pars{n-1-k}!} \pars{\frac{n}{\pars{n-k} \pars{k}}} \\
                                    &= \frac{\pars{n \pars{n-1}!}}{\pars{k \pars{k-1}!} \pars{\pars{n-k}\pars{n-1-k}!}} \\
                                    &= \frac{n!}{k! \pars{n-k}!} \\
                                    &= \binom{n}{k} \\
\end{align*}

\subproblemat{}

\textbf{(Leibniz Rule).}
$\dagger$ Suppose $f$ and $g$ are differentiable functions.
We know from high school calculus the product rule: \[(fg)'=f'g+fg'\]
Prove using induction the following Leibniz Rule: \[(fg)^{(n)}= \sum_{k=0}^n \binom{n}{k}f^{(n-k)}g^{(k)}\]
Here $f^{(n)}$ denotes the $n$th derivative of $f$, with the convention that $f^{(0)}=f$.
[Hint: use part (a).]

\hrule

We proceed using induction.
For the $n=1$ case, we have that $\pars{fg}' = f'g + g'f$, which is consistent with the product rule.

We then assume that statement is true for the $n$th derivative and prove it for the $n+1$th derivative.
The $n+1$th derivative of $fg$ can be written as, using the product rule,

\begin{align*}
  \pars{fg}^{\pars{n+1}} &= \pars{\pars{fg}^{\pars{n}}}' \\
                         &= \pars{\sum \limits_{k=0}^n \binom{n}{k}f^{(n-k)}g^{(k)}} ' \\
                         &= \sum \limits_{k=0}^n \binom{n}{k} f^{\pars{n-k+1}} g^{\pars{k}} + \sum \limits_{k=0}^n \binom{n}{k} f^{\pars{n-k}} g^{\pars{k+1}}
\end{align*}

From here, we alter the indices $k$ to move us towards Pascal's identity.
The second term's index can be altered to be more amenable to Pascal's identity by scaling $k$ up 1 in the bounds of summation and scaling $k$ down 1 in the summand, as follows,
$$
\sum \limits_{k=0}^n \binom{n}{k} f^{\pars{n-k+1}} g^{\pars{k}} + \sum \limits_{k=1}^{n+1} \binom{n}{k-1} f^{\pars{n-k+1}} g^{\pars{k}}
$$

We then consider the first term ($k=0$) of the first sum and the last term ($k=n+1$) of the second sum individually to be able to factor out the resultant.

\begin{align*}
  \sum \limits_{k=0}^n \binom{n}{k} f^{\pars{n-k+1}} g^{\pars{k}} + \sum \limits_{k=1}^{n+1} \binom{n}{k-1} f^{\pars{n-k+1}} g^{\pars{k}} &= \binom{n}{0} f^{\pars{n+1}} g^{\pars{0}} + \binom{n}{n} f^{\pars{0}} g^{\pars{n+1}} + \sum \limits_{k=1}^n \binom{n}{k} f^{\pars{n-k+1}} g^{\pars{k}} \\
                                                                                                                                          &+ \sum \limits_{k=1}^{n} \binom{n}{k-1} f^{\pars{n-k+1}} g^{\pars{k}}
\end{align*}

This is a very long and complicated expression, but we can simplify it a lot.
We first focus on the sums only and note that they have the same product of derivatives and summation bounds.
This makes it possible to combine them and distribute out their common factors.
We can then combine the binomial coefficients using Pascal's identity.

\begin{align*}
  \sum \limits_{k=1}^n \binom{n}{k} f^{\pars{n-k+1}} g^{\pars{k}} + \sum \limits_{k=1}^{n} \binom{n}{k-1} f^{\pars{n-k+1}} g^{\pars{k}} &= \sum \limits_{k=1}^{n} f^{\pars{n-k+1}} g^{\pars{k}} \pars{\binom{n}{k} + \binom{n}{k-1}} \\
  &= \sum \limits_{k=1}^{n} f^{\pars{n-k+1}} g^{\pars{k}} \binom{n+1}{k} \\
\end{align*}

Combining this with the indiviual terms yields,

$$
  \binom{n}{0} f^{\pars{n+1}} g + \binom{n}{n} f g^{\pars{n+1}} + \sum \limits_{k=1}^{n} \binom{n+1}{k} f^{\pars{n-k+1}} g^{\pars{k}}
$$

We can then incorporate the individual terms into the sum, by extending the bounds by 1 both ways, resulting in

$$
\sum \limits_{k=0}^{n+1} \binom{n+1}{k} f^{\pars{n-k+1}} g^{\pars{k}} 
$$

\end{document}